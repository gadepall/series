\begin{enumerate}[label=\thesubsection.\arabic*,ref=\thesubsection.\theenumi]

\item Suppose four positive numbers $a_{1},a_{2},a_{3},a_{4}$ are in GP. Let $b_{1} = a_{1}, b_{2} = b_{1} + a_{2}, b_{3} = b_{2} + a_{3}$ and $b_{4} = b_{3} + a_{4}$.

STATEMENT-1 : The numbers $b_{1},b_{2},b_{3},b_{4}$ are neither in AP nor in GP and STATEMENT-2 : The numbers $b_{1},b_{2},b_{3},b{4}$ are in H.P.\hfill(2008)

\begin{enumerate}
    

\item STATEMENT-1 is True, STATEMENT-2 is True; STATEMENT-2 is the correct explanation for STATEMENT-1

\item STATEMENT-1 is True, STATEMENT-2 is True;

STATEMENT-2 is NOT a correct explanation for STATEMENT-1

\item STATEMENT-1 is True, STATEMENT-2 is False

\item STATEMENT-1 is False, STATEMENT-2 is True
\end{enumerate}
		  \item Let the harmonic mean and geometric mean of two positive numbers be the ratio $4:5$. Then the two numbers are in 
			  ratio \dots\hfill{(1992)}
          

   \item Let $a,b,c$ be positive integers such that $\frac{b}{a}$ is an integer. If $a$,$b,c$ are in geometric progression and the arithmetic mean of $a,b,c$ is $b + 2$, then the value of $\frac{a^{2} + a - 14}{a + 1}$ is \rule{1cm}{0.1pt}.
	   \hfill(2014)

\item {$l, m, n$ are the $p^{th}, q^{th}$ and $r^{th}$ term of a GP all positive, then $\mydet{\log l & p & 1 \\ \log m & q & 1 \\ \log n & r & 1 }$ equals}

	{\hfill{\brak{2002}}} 
\begin{enumerate}
\begin{multicols}{4}
\item{$1$}
\item{$2$}
\item{$1$}
\item{$0$}
\end{multicols}
\end{enumerate}
\item {Fifth term of a GP is $2$, then the product of its $9$ terms is}
{\hfill{\brak{2002}}}
\begin{enumerate}	
\begin{multicols}{4}
\item  {$256$}
\item  {$512$}
\item  {$1024$}
\item  {none of these}
\end{multicols}
\end{enumerate}
    \item  The first two terms of a geometric progression add up to $12$. The sum of the third and the fourth terms is $48$. If the terms of the geometric progression are alternately positive and negative,  then the first term is   
    \hfill(2008)
%    
    \begin{multicols}{4}
\begin{enumerate}    
    \item$-4$
    \item$-12$
    \item$12$ 
    \item$4$ 
    \end{enumerate}
\end{multicols}
    \item If $m$ is the AM of two distinct real numbers $l$ and $n$ $(l, n>1)$ and $G_1$,  $G_2$ and $G_3$ are three geometric means between $l$ and $n$,  then $G_1^4+2G_2^4+G_3^4$ equals 
%    
    \hfill(2015)
    \begin{multicols}{4}
\begin{enumerate}    
    \item$4lmn^2$
    \item$4l^2m^2n^2$
    \item$4l^2mn$
    \item$4lm^2n$ 
    \end{enumerate}
\end{multicols}
%
    \item If the $2^{nd}$,  $5^{th}$ and $9^{th}$ terms of a non-constant AP are in GP,  then the common ratio of this GP is
    \hfill(2016)
    \begin{multicols}{4}
\begin{enumerate}    
    \item $1$
    \item $\frac{7}{4}$
    \item $\frac{8}{5}$
    \item $\frac{4}{3}$
    \end{enumerate}
\end{multicols}
%    
	\item  The third term of a geometric progression is $4$. The product of five terms is\hfill{(1982)}
\begin{multicols}{4}
\begin{enumerate}    
     \item${4^3}$ 
    \item${4^5}$
    \item${4^4}$
    \item none of these
\end{enumerate}
\end{multicols}
\item Consider an infinite geometric series with first term $a$ and common ratio $r$. If its sum is $4$ and the second term is $3/4$,  then \hfill (2000)
        \begin{multicols}{4}
\begin{enumerate}    
            \item $a=4/7, r=3/7$
            \item $a=2, r=3/8$
            \item $a=3/2, r=1/2$
            \item $a=3, r=1/4$
            \end{enumerate}
            \end{multicols}
\item[]  
%
Let $A_{1},  G_{1},  H_{1} $ denote the arithmetic,  geometric and harmonic means,  respectively,  of two distinct positive numbers. For $n\geq 2$,  Let $A_{n-1}$ and $H_{n-1}$ have arithmetic,  geometric and harmonic means as $A_{n}, G_{n}, H_{n}$ respectively. \hfill(2007)
\begin{enumerate}    
%      
%  
 \item Which one of the following statements is correct ?
\begin{enumerate}    
%
% 
	\item$G_{1}>G_{2}>G_{3}>\dots$ 
%
 \item$G_{1}<G_{2}<G_{3}<\dots$
%
\item$G_{1}=G_{2}=G_{3}=\dots$
%
\item$G_{1}<G_{3}<G_{5}<\dots$ and $G_{2}>G_{4}>G_{6}>\dots$
\end{enumerate}
%
\item Which one of the following statements is correct ?
\begin{enumerate}    
%    
\item $A_{1}>A_{2}>A_{3}>\dots$ 
%
\item $A_{1}<A_{2}<A_{3}<\dots$
%
\item $A_{1}>A_{3}>A_{5}>\dots$ and $A_{2}<A_{4}<A_{6}<\dots$
%
\item $A_{1}<A_{3}<A_{5}<\dots$ and $A_{2}>A_{4}>A_{6}>\dots$
\end{enumerate}
%
\item Which one of the following statements is correct ?
\begin{enumerate}    
%
	\item $H_{1}>H_{2}>H_{3}>\dots$ 
%
 \item $H_{1}<H_{2}<H_{3}<\dots$
%
\item $H_{1}>H_{3}>H_{5}>\dots$ and $H_{2}<H_{4}<H_{6}<\dots$
%
\item $H_{1}<H_{3}<H_{5}<\dots$ and $H_{2}>H_{4}>H_{6}>\dots$
\end{enumerate}
\end{enumerate}
%
    \item The harmonic mean of two numbers is 4. Their arithmetic mean $A$ and the geometric mean $G$ satisfy the relation
    $2A + G^2 = 27$.
    Find the two numbers.  \hfill\brak{1979}
    \item Does there exist a geometric progression containing 27,  8 and 12 as three of its terms? If it exits,  how many such progressions are possible?  \hfill\brak{1982}
      \item Let $ a_{1}, a_{2}, \dots, a_{n} $ be positive real numbers in geometric progression. For each $n$, let $ A_{n}, G_{n}, H_{n} $ be respectively,  the arithmetic mean,  geometric mean and harmonic mean of $ a_{1}, a_{2}, \dots, a_{n}.$ Find an expression for the geometric mean of $ G_{1}, G_{2}, \dots, G_{n} $ in terms of $ A_{1}, A_{2}, \dots, A_{n}, H_{1}, H_{2}, \dots, H_{n}.$ 
%      
	      \hfill \brak{2001 }                              
%       
       \item Let $a, b$ be positive real numbers. If $ a, A_{1}, A_{2}, b $ are in arithmetic progression,  $ a, G_{1}, G_{2}, b $ are in geometric progression and $ a, H_{1}, H_{2}, b $ are in harmonic progression,  show that 
	       $$ \frac{G_{1}G_{2}}{H_{1}H_{2}}=\frac{A_{1}+A_{2}}{H_{1}+H_{2}}=\frac{\brak{2a+b}\brak{a+2b}}{9ab} $$ 
%
		\hfill \brak{2002 }                             
	\item If $a, b, c$ are in AP, $a^{2}, b^{2}, c^{2} $ are in HP,  then prove that either $ a=b=c $ or $a, b, -\frac{c}{2}$ form a GP.
%		                    
		\hfill \brak{2003 }                            
    \item Three positive numbers form an increasing GP. If the middle term in this GP is doubled,  the new numbers are in AP. Then the common ratio of the GP is 
%    
    \hfill(2014)
    \begin{multicols}{4}
\begin{enumerate}    
    \item$2-\sqrt{3}$
    \item$2+\sqrt{3}$
    \item$\sqrt{2}+\sqrt{3}$
    \item$3+\sqrt{2}$ 
    \end{enumerate}
\end{multicols}
%
  \item{If $a,  b$ and $c$ be three distinct real numbers in GP and $a+b+c=xb$,  then $x$ cannot be \hfill{(2019)}
\begin{multicols}{4}
\begin{enumerate}    
  \item {-2} \item{4}
  \item{-3}
  \item{2}
  \end{enumerate}
\end{multicols}}
	\item If the first and the $\brak{2n-1}^{th}$ terms of an AP,  a GP and an HP are equal and their $n^{th}$ terms are $a,  b \;  \text{and} \;  c$ respectively,  then \hfill\brak{1988}
%
\begin{multicols}{4}
\begin{enumerate}    
\item $a=b=c$
\item $a \geq b \geq c$
\item $a+b=c$
\item $ac-b^2=0$
\end{enumerate}
\end{multicols}
%
\item If $x>1, y>1, z>1$ are in GP, then $\frac{1}{1+\ln x}, \frac{1}{1+\ln y}, \frac{1}{1+\ln z}$ are in 
\hfill\brak{1998}
\begin{multicols}{4}
\begin{enumerate}    
\item AP
\item HP
\item GP
\item None of these
\end{enumerate}
\end{multicols}
%
\item  If $x$, $y$ and $z$ are $p^{th}, q^{th}$ and $r^{th}$ terms respectively of an AP and also of a GP,  then ${x^{y-z} y^{z-x} z^{x-y}}$ 
		    is equal to \hfill{(1982)}
\begin{multicols}{4}
\begin{enumerate}     
  \item $xyz$ 
  \item $0$
  \item $1$ 
  \item none of these
  \end{enumerate}
\end{multicols}
%
\item If $\log_e(a+c), \log_e(a-c), \log_e(a-2b+c)$ are in AP, then \hfill (1994)
    \begin{multicols}{2}
%        
\begin{enumerate}    
        \item $a, b, c$ are in AP
        \item $a^2, b^2, c^2$ are in AP
        \item $a, b, c$ are in GP
        \item $a, b, c$ are in HP
    \end{enumerate}
    \end{multicols}
\item Let $\alpha$, $\beta$ be the roots of $x^2-x+p=0$ and $\gamma$, $\delta$ be the roots of $x^2-4x+q=0$. If $\alpha$, $\beta$, $\gamma$, $\delta$ are in GP, then the integral values of $p$ and $q$ respectively are \hfill(2001)
            \begin{multicols}{4}
\begin{enumerate}    
                \item $-2, -32$
                \item $-2, 3$
                \item $-6, 3$
                \item $6, -32$
%        
    \end{enumerate}
    \end{multicols}
\item Let the positive numbers $a, b, c, d$ be in AP. Then $abc, abd, acd, bcd$ are \hfill(2001)
    \begin{multicols}{2}
\begin{enumerate}    
        \item NOT in AP/GP/H.P
        \item in AP
        \item in GP
        \item  in HP
        \end{enumerate}
\end{multicols}
\item Suppose $a, b, c$ are in AP and $a^2, b^2, c^2$ are in GP if $a<b<c$ and $a+b+c=3/2$,  then the value of $a$ is \hfill(2002)
            \begin{multicols}{4}
\begin{enumerate}    
             \item $\frac{1}{2\sqrt{2}}$
             \item $\frac{1}{2\sqrt{3}}$
             \item $\frac{1}{2}-\frac{1}{\sqrt{3}}$
             \item $\frac{1}{2}-\frac{1}{\sqrt{2}}$
            \end{enumerate}
            \end{multicols}
\item An infinite GP has first term $x$ and sum $5$ then $x$ belongs to \hfill(2004)
            \begin{multicols}{4}
\begin{enumerate}    
                \item $x<-10$
                \item $-10<x<0$
                \item $0<x<10$
                \item $x>10$
                \end{enumerate}
                \end{multicols}
\item In the quadratic equation $ax^2+bx+c=0, \triangle=b^2-4ac$ and $\alpha$+$\beta$, $\alpha^2$+$\beta^2$, $\alpha^3$+$\beta^3$ are in GP where $\alpha$, $\beta $ are roots of $ax^2+bx+c=0$, then \hfill(2005)
                \begin{multicols}{4}
\begin{enumerate}    
                    \item $\triangle\neq0$
                    \item $b\triangle=0$
                    \item $c\triangle=0$
                    \item $\triangle=0$
                \end{enumerate}
                \end{multicols}
    \item Find three numbers $a, b, c$ between 2 and 18 such that
\begin{enumerate}    
    \item their sum is 25
    \item the numbers $2, a, b$ are consecutive terms of an AP
    \\
    and
    \item the numbers $b, c, 18$ are consecutive terms of a GP\hfill\brak{1983}
    \end{enumerate}
%  
      \item  Let $ a, b, c, d $ be real numbers in GP. If $u, v, w$ satisfy the system of equations  
%    
	      \hfill \brak{1999 }
%      
\begin{align*}
	u+2v+3w&=6  
	  \\
	  4u+5v+6w&=12 
	  \\
	  6u+9v&=4 
\end{align*}
      then show that the roots of the equations 
		$$\brak{\frac{1}{u}+\frac{1}{v}+\frac{1}{w}}x^{2}+\sbrak{\brak{b-c}^{2}+\brak{c-a}^{2}+\brak{d-b}^{2}}x+u+v+w=0 $$ and $$ 20x^{2}+10\brak{a-d}^{2}x-9=0 $$ are reciprocals of each other.
 \item For any three positive real numbers $a, b$ and $c$,  $$9(25a^2+b^2)+ 25(c^2-3ac) = 15b(3a+c).$$ Then \hfill (2017)
	 \begin{multicols}{2}
\begin{enumerate}    
  \item$a, b$ and $c$ are in AP
  \item{$b, c$ and $a$ are in GP}
  \columnbreak
  \item $b, c$ and $a$ are in AP
  \item{$a, b$ and $c$ are in GP}
  \end{enumerate}
  \end{multicols}
\item       Let $a_1, a_2, a_3, \dots $ be a sequence of positive integers in arithmetic progression with common difference 2. Also, let $b_1, b_2, b_3, \dots $ be a sequence of positive integers in geometric progression with common ratio 2. If $a_1 = b_1 = c$, then the number of all possible values of $c$ for which the equality 

\begin{align*}
     2(a_1 + a_2 + \dots + a_n) = b_1 + b_2 + \dots + b_n
\end{align*}
    holds for some positive integer $n$, is \rule{1cm}{0.1pt}.

\hfill \brak{2020}
\end{enumerate}
