\begin{enumerate}[label=\thesubsection.\arabic*,ref=\thesubsection.\theenumi]

\item {Fifth term of a GP is $2$, then the product of its $9$ terms is}
{\hfill{\brak{2002}}}
\begin{enumerate}	
\begin{multicols}{4}
\item  {$256$}
\item  {$512$}
\item  {$1024$}
\item  {none of these}
\end{multicols}
\end{enumerate}
    \item  The first two terms of a geometric progression add up to $12$. The sum of the third and the fourth terms is $48$. If the terms of the geometric progression are alternately positive and negative,  then the first term is   
    \hfill(2008)
%    
    \begin{multicols}{4}
\begin{enumerate}    
    \item$-4$
    \item$-12$
    \item$12$ 
    \item$4$ 
    \end{enumerate}
\end{multicols}
	\item  The third term of a geometric progression is $4$. The product of five terms is\hfill{(1982)}
\begin{multicols}{4}
\begin{enumerate}    
     \item${4^3}$ 
    \item${4^5}$
    \item${4^4}$
    \item none of these
\end{enumerate}
\end{multicols}
\item Consider an infinite geometric series with first term $a$ and common ratio $r$. If its sum is $4$ and the second term is $3/4$,  then \hfill (2000)
        \begin{multicols}{4}
\begin{enumerate}    
            \item $a=4/7, r=3/7$
            \item $a=2, r=3/8$
            \item $a=3/2, r=1/2$
            \item $a=3, r=1/4$
            \end{enumerate}
            \end{multicols}
    \item The harmonic mean of two numbers is 4. Their arithmetic mean $A$ and the geometric mean $G$ satisfy the relation
    $2A + G^2 = 27$.
    Find the two numbers.  \hfill\brak{1979}
    \item Does there exist a geometric progression containing 27,  8 and 12 as three of its terms? If it exits,  how many such progressions are possible?  \hfill\brak{1982}
    \item Three positive numbers form an increasing GP. If the middle term in this GP is doubled,  the new numbers are in AP. Then the common ratio of the GP is 
%    
    \hfill(2014)
    \begin{multicols}{4}
\begin{enumerate}    
    \item$2-\sqrt{3}$
    \item$2+\sqrt{3}$
    \item$\sqrt{2}+\sqrt{3}$
    \item$3+\sqrt{2}$ 
    \end{enumerate}
\end{multicols}
%
\item An infinite GP has first term $x$ and sum $5$ then $x$ belongs to \hfill(2004)
            \begin{multicols}{4}
\begin{enumerate}    
                \item $x<-10$
                \item $-10<x<0$
                \item $0<x<10$
                \item $x>10$
                \end{enumerate}
                \end{multicols}
    \item Find three numbers $a, b, c$ between 2 and 18 such that
\begin{enumerate}    
    \item their sum is 25
    \item the numbers $2, a, b$ are consecutive terms of an AP
    \\
    and
    \item the numbers $b, c, 18$ are consecutive terms of a GP\hfill\brak{1983}
    \end{enumerate}
%  
\item Let $a_1, a_2, a_3 $ \dots be a geometric progression such that $a_1 < 0$, $a_1 + a_2 = 4$ and $a_3 + a_4 = 16$. If\begin{align*}\sum_{i=1}^{9} a_i = 4\lambda,\end{align*}then $\lambda$ is equal to
\hfill\brak{2020}
\begin{multicols}{4}
\begin{enumerate}
   \item -171
   \item 171
   \item $\frac{511}{3}$
   \item -513
\end{enumerate}
\end{multicols}
\end{enumerate}
