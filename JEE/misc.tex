\begin{enumerate}[label=\thesubsection.\arabic*,ref=\thesubsection.\theenumi]

\item Let $a_{1},  a_{2},  a_{3}\dots a_{100}$ be an  AP  with $a_{1}= 3$ and $$S_{p} =\sum\limits_{i=1}^{p} a_{i}, 1\leq p\leq 100.$$ For any integer $n$ with $1 \leq n \leq 20, $ let $m= 5n$. If $\frac{S_{m}}{S_{n}}$ does not depend on $n$,  then $a_{2}$ is \rule{1cm}{0.1pt}.\hfill(2011)

		  \item  Let $p$ and $q$ be the roots of the equation                    $${x^2-2x+A=0}$$ and $r$ and $s$ be the roots of the                     equation ${x^2-18x+B=0}$. If ${p<q<r<s}$ are                                      in  AP ,  then find $A$ and $B$. \hfill{(1977)}
	\item {If $ 1,  \log_9 \brak{3^{1-x} +2},  \log_3 \brak{4\cdot3^x -1}$ are in AP then $x$ equals}
{\hfill{\brak{2002}}}
\begin{multicols}{4}
\begin{enumerate}    
\item  {$\log_3 4$}
 \item {$1-\log_3 4$}
 \item {$1-\log_4 3$}
 \item {$\log_4 3$}
\end{enumerate}
\end{multicols}
%
\item {Let $T_r$ be the $r^{th}$ term of an AP whose first term is a and common difference is $d$. If for some positive integers $m, n,  m\neq n,  T_m = \frac{1}{n}$ and $T_n = \frac{1}{m}$,  then $a-d$ equals} 
{\hfill{\brak{2004}}}
\begin{multicols}{4}
\begin{enumerate}    
\item  {$\frac{1}{m}+\frac{1}{n}$}
\item  {$1$}
\item  {$\frac{1}{mn}$}
\item  {$0$}
\end{enumerate}
\end{multicols}
\item Let $a_1,  a_2,  a_3 \dots$ be terms of an AP. If $\frac{a_1+a_2+\dots a_p}{a_1+a_2+\dots a_q}= \frac{p^2}{q^2},  p \neq q$,  then $\frac{a_6}{a_{21}}$ equals
\hfill \brak{2006}
\begin{multicols}{4}
\begin{enumerate}    
\item  {$\frac{41}{11}$}
\item  {$\frac{7}{2}$}
\item  {$\frac{2}{7}$}
\item  {$\frac{11}{41}$}
\end{enumerate}
\end{multicols}
    \item If $a_1, a_2, \dots, a_n$ are in HP,  then the expression $a_1a_2+a_2a_3+\dots+a_{n-1}a_n$ is equal to

%    
    \hfill(2006)
%
    \begin{multicols}{4}
\begin{enumerate}    
    \item$n(a_1-a_n)$
    \item$(n-1)(a_1-a_n)$
    \item$na_1a_n$
    \item$(n-1)a_1a_n$ 
    \end{enumerate}
\end{multicols}
%    
%
  \item Let $a, b, c \in R$. If $f(x)=ax^2+bx+c$ is such that $a+b+c=3$ and $$f(x+y)=f(x)+f(y) \forall x, y \epsilon R,$$  then $\sum _{n=1}^{10}  f(n)$  is  equal  to \hfill (2017)
	 \begin{multicols}{4}
\begin{enumerate}    
  \item{255}
  \item{330}
  \item{165}
  \item{190}
  \end{enumerate}
  \end{multicols}
\item Let $T_r$ be the $r^{th}$ term of an AP,  for $r=1, 2, 3, \dots$ If for some positive integers $m, n$ we have
$T_m=\frac{1}{n}$ and $T_n=\frac{1}{m}$ , then $T_{mn}$ equals \hfill\brak{1998}
%
\begin{multicols}{4}
\begin{enumerate}    
\item $\frac{1}{mn}$
\item $\frac{1}{m} + \frac{1}{m}$
\item $1$
\item $0$
\end{enumerate}
\end{multicols}
\item Let ${a_1, a_2, a_3, \dots}$ be in harmonic progression with ${a_1}=5$ and ${a_{20}}=25$. The least positive integer $n$ for which ${a_n<0}$ is \hfill(2012)
                \begin{multicols}{4}
\begin{enumerate}    
                    \item 22
                    \item 23
                    \item 24
                    \item 25
                    \end{enumerate}
                    \end{multicols}
\item Let ${b_i}>1$ for $i=1, 2, \dots, 101$. Suppose ${\log_e}{b_1}, {\log_e}{b_2}, \dots, {\log_e}{b_{101}}$ are in  AP   with the common difference ${\log_e}2$. Suppose ${a_1, a_2, \dots, a_{101}}$ are in AP such that ${a_1=b_1}$ and ${a_{51}=b_{51}}$. If $t={b_1+b_2+\dots+b_{51}}$ and $s={a_1+a_2+\dots+a_{53}}$,  then \hfill ( 2016)
                    \begin{multicols}{2}
\begin{enumerate}    
%                    
                        \item $s>t$ and ${a_{101}>b_{101}}$
                        \item $s>t$ and ${a_{101}<b_{101}}$
                        \item $s<t$ and ${a_{101}>b_{101}}$
                        \item $s<t$ and ${a_{101}<b_{101}}$
                        \end{enumerate}
                        \end{multicols}
%    
\item[]
 Let $ V_{r} $ denote the sum of first $r$ terms of an  AP   whose first term is $r$ and the common difference is $\brak{2r-1}$. Let $ T_{r}=V_{r+1}-V_{r}-2 $ and $ Q_{r}=T_{r+1}-T_{r}$ for $r=1, 2, \dots$
% 
%
 \item The sum  $  V_{1}+V_{2}+\dots+V_{n} $  is 
% 
	 \hfill \brak{2007 }                            
     \begin{multicols}{2}
\begin{enumerate}    
%         
	     \item $\frac{1}{12}n\brak{n+1}\brak{3n^{2}-n+1}$
	     \item $\frac{1}{12}n\brak{n+1}\brak{3n^{2}+n+2}$
	     \item $\frac{1}{2}n\brak{2n^{2}-n+1}$
	     \item $\frac{1}{3}\brak{2n^{3}-2n+3}$
    \end{enumerate}
\end{multicols} 
%
  \item $T_{r}$ is always 
%                           
	  \hfill \brak{2007 }                    
                  \begin{multicols}{2}      
\begin{enumerate}     
%	
       \item an odd number 
       \item an even number
	\item a prime number 
        \item composite number
%
	  \end{enumerate}
   \end{multicols}
    \item Which one of the following is a correct statement? 
%          
	    \hfill \brak{2007 }                                  
\begin{enumerate}    
	\item $Q_{1}, Q_{2}, Q_{3}, \dots$ are in AP with common difference 5 
	\item $Q_{1}, Q_{2}, Q_{3}, \dots$ are in AP with common difference 6
	\item $Q_{1}, Q_{2}, Q_{3}, \dots$ are in AP with common difference 11
	\item $Q_{1}=Q_{2}=Q_{3}=\dots$
	\end{enumerate}
%       
%
%	    
	     \item If $ \log_{3}{2}, \log_{3}{2^{x}-3} $ and $ \log_{3}{\brak{2^{x}-\frac{7}{2}}} $ are in  AP , determine the value of $x$.  
%     
	      \hfill \brak{1990 }
%      
%
      \item Let $p$ be the first of $n$ arithmetic means between two numbers and $q$ the first of $n$ harmonic means between the same numbers . Show that $q$ does not lie between $p$ and $\brak{\frac{n+1}{n-1}}^{2}p$ 
%       
	      \hfill \brak{1991 }
%      
%
%       
%
		\item  The real numbers $ x_{1}, x_{2}, x_{3} $ satisfying the equation $ x^{3}-x^{2}+\beta x+\gamma=0 $ are in AP. Find the intervals in which $ \beta $ and $\gamma$ lie.
%       
			\hfill \brak{1996 }
%      
%
%
      \item The fourth power of the common difference of an  AP  with integer entries is added to the product of any four consecutive terms of it. Prove that the resulting sum is the square of an integer.
%      
	      \hfill \brak{2000 }
\item     Let $p,q$ amd $r$  be nonzero real numbers that are the $10^{th}, 100^{th},$ and $1000^{th}$ terms of a harmonic progression, respectively. Consider the following system of linear equations

	\hfill \brak{2022}
\begin{align*}
x + y + z &= 1
\\
10x + 100y + 1000z &= 0
\\
qr x + pr y + pq z &= 0
\end{align*}

		\begin{multicols}{2}
			\begin{enumerate}[label=(\Roman*)]		
%\begin{enumerate}		
\item     If $ \frac{q}{r} = 10 $, then the system of linear equations has 

\item      If $ \frac{p}{r} \neq 100 $, then the system of linear equations has 

\item      If $ \frac{p}{q} \neq 10 $, then the system of linear equations has 
\item      If $ \frac{p}{q} = 10 $, then the system of linear equations has 
\end{enumerate}
\columnbreak
					\begin{enumerate}[label=(\Alph*)]		
\item     $ x = 0,  y = \frac{10}{9}, z = -\frac{1}{9} $ as a solution  
\item     $ x = \frac{10}{9},  y = -\frac{1}{9},  z = 0 $ as a solution 
\item     infinitely many solutions                               
\item     no solution 
\end{enumerate}
                                         \end{multicols} 

\begin{enumerate}		
\item     $(I)\to(T);(II)\to(C);(III)\to(D);(IV)\to(T)$
\item     $(I)\to(B);(II)\to(D);(III)\to(D);(IV)\to(C)$   
\item     $(I)\to(B);(II)\to(C);(III)\to(A);(IV)\to(C)$
\item     $(I)\to(T);(II)\to(D);(III)\to(A);(IV)\to(T)$
\end{enumerate}
\item Suppose four positive numbers $a_{1},a_{2},a_{3},a_{4}$ are in GP. Let $b_{1} = a_{1}, b_{2} = b_{1} + a_{2}, b_{3} = b_{2} + a_{3}$ and $b_{4} = b_{3} + a_{4}$.

STATEMENT-1 : The numbers $b_{1},b_{2},b_{3},b_{4}$ are neither in AP nor in GP and STATEMENT-2 : The numbers $b_{1},b_{2},b_{3},b{4}$ are in H.P.\hfill(2008)

\begin{enumerate}
    

\item STATEMENT-1 is True, STATEMENT-2 is True; STATEMENT-2 is the correct explanation for STATEMENT-1

\item STATEMENT-1 is True, STATEMENT-2 is True;

STATEMENT-2 is NOT a correct explanation for STATEMENT-1

\item STATEMENT-1 is True, STATEMENT-2 is False

\item STATEMENT-1 is False, STATEMENT-2 is True
\end{enumerate}
		  \item Let the harmonic mean and geometric mean of two positive numbers be the ratio $4:5$. Then the two numbers are in 
			  ratio \dots\hfill{(1992)}
          

   \item Let $a,b,c$ be positive integers such that $\frac{b}{a}$ is an integer. If $a$,$b,c$ are in geometric progression and the arithmetic mean of $a,b,c$ is $b + 2$, then the value of $\frac{a^{2} + a - 14}{a + 1}$ is \rule{1cm}{0.1pt}.
	   \hfill(2014)

\item {$l, m, n$ are the $p^{th}, q^{th}$ and $r^{th}$ term of a GP all positive, then $\mydet{\log l & p & 1 \\ \log m & q & 1 \\ \log n & r & 1 }$ equals}

	{\hfill{\brak{2002}}} 
\begin{enumerate}
\begin{multicols}{4}
\item{$1$}
\item{$2$}
\item{$1$}
\item{$0$}
\end{multicols}
\end{enumerate}
    \item If $m$ is the AM of two distinct real numbers $l$ and $n$ $(l, n>1)$ and $G_1$,  $G_2$ and $G_3$ are three geometric means between $l$ and $n$,  then $G_1^4+2G_2^4+G_3^4$ equals 
%    
    \hfill(2015)
    \begin{multicols}{4}
\begin{enumerate}    
    \item$4lmn^2$
    \item$4l^2m^2n^2$
    \item$4l^2mn$
    \item$4lm^2n$ 
    \end{enumerate}
\end{multicols}
%
    \item If the $2^{nd}$,  $5^{th}$ and $9^{th}$ terms of a non-constant AP are in GP,  then the common ratio of this GP is
    \hfill(2016)
    \begin{multicols}{4}
\begin{enumerate}    
    \item $1$
    \item $\frac{7}{4}$
    \item $\frac{8}{5}$
    \item $\frac{4}{3}$
    \end{enumerate}
\end{multicols}
%    
\item[]  
%
Let $A_{1},  G_{1},  H_{1} $ denote the arithmetic,  geometric and harmonic means,  respectively,  of two distinct positive numbers. For $n\geq 2$,  Let $A_{n-1}$ and $H_{n-1}$ have arithmetic,  geometric and harmonic means as $A_{n}, G_{n}, H_{n}$ respectively. \hfill(2007)
\begin{enumerate}    
%      
%  
 \item Which one of the following statements is correct ?
\begin{enumerate}    
%
% 
	\item$G_{1}>G_{2}>G_{3}>\dots$ 
%
 \item$G_{1}<G_{2}<G_{3}<\dots$
%
\item$G_{1}=G_{2}=G_{3}=\dots$
%
\item$G_{1}<G_{3}<G_{5}<\dots$ and $G_{2}>G_{4}>G_{6}>\dots$
\end{enumerate}
%
\item Which one of the following statements is correct ?
\begin{enumerate}    
%    
\item $A_{1}>A_{2}>A_{3}>\dots$ 
%
\item $A_{1}<A_{2}<A_{3}<\dots$
%
\item $A_{1}>A_{3}>A_{5}>\dots$ and $A_{2}<A_{4}<A_{6}<\dots$
%
\item $A_{1}<A_{3}<A_{5}<\dots$ and $A_{2}>A_{4}>A_{6}>\dots$
\end{enumerate}
%
\item Which one of the following statements is correct ?
\begin{enumerate}    
%
	\item $H_{1}>H_{2}>H_{3}>\dots$ 
%
 \item $H_{1}<H_{2}<H_{3}<\dots$
%
\item $H_{1}>H_{3}>H_{5}>\dots$ and $H_{2}<H_{4}<H_{6}<\dots$
%
\item $H_{1}<H_{3}<H_{5}<\dots$ and $H_{2}>H_{4}>H_{6}>\dots$
\end{enumerate}
\end{enumerate}
%
      \item Let $ a_{1}, a_{2}, \dots, a_{n} $ be positive real numbers in geometric progression. For each $n$, let $ A_{n}, G_{n}, H_{n} $ be respectively,  the arithmetic mean,  geometric mean and harmonic mean of $ a_{1}, a_{2}, \dots, a_{n}.$ Find an expression for the geometric mean of $ G_{1}, G_{2}, \dots, G_{n} $ in terms of $ A_{1}, A_{2}, \dots, A_{n}, H_{1}, H_{2}, \dots, H_{n}.$ 
%      
	      \hfill \brak{2001 }                              
%       
       \item Let $a, b$ be positive real numbers. If $ a, A_{1}, A_{2}, b $ are in arithmetic progression,  $ a, G_{1}, G_{2}, b $ are in geometric progression and $ a, H_{1}, H_{2}, b $ are in harmonic progression,  show that 
	       $$ \frac{G_{1}G_{2}}{H_{1}H_{2}}=\frac{A_{1}+A_{2}}{H_{1}+H_{2}}=\frac{\brak{2a+b}\brak{a+2b}}{9ab} $$ 
%
		\hfill \brak{2002 }                             
	\item If $a, b, c$ are in AP, $a^{2}, b^{2}, c^{2} $ are in HP,  then prove that either $ a=b=c $ or $a, b, -\frac{c}{2}$ form a GP.
%		                    
		\hfill \brak{2003 }                            
  \item{If $a,  b$ and $c$ be three distinct real numbers in GP and $a+b+c=xb$,  then $x$ cannot be \hfill{(2019)}
\begin{multicols}{4}
\begin{enumerate}    
  \item {-2} \item{4}
  \item{-3}
  \item{2}
  \end{enumerate}
\end{multicols}}
	\item If the first and the $\brak{2n-1}^{th}$ terms of an AP,  a GP and an HP are equal and their $n^{th}$ terms are $a,  b \;  \text{and} \;  c$ respectively,  then \hfill\brak{1988}
%
\begin{multicols}{4}
\begin{enumerate}    
\item $a=b=c$
\item $a \geq b \geq c$
\item $a+b=c$
\item $ac-b^2=0$
\end{enumerate}
\end{multicols}
%
\item If $x>1, y>1, z>1$ are in GP, then $\frac{1}{1+\ln x}, \frac{1}{1+\ln y}, \frac{1}{1+\ln z}$ are in 
\hfill\brak{1998}
\begin{multicols}{4}
\begin{enumerate}    
\item AP
\item HP
\item GP
\item None of these
\end{enumerate}
\end{multicols}
%
\item  If $x$, $y$ and $z$ are $p^{th}, q^{th}$ and $r^{th}$ terms respectively of an AP and also of a GP,  then ${x^{y-z} y^{z-x} z^{x-y}}$ 
		    is equal to \hfill{(1982)}
\begin{multicols}{4}
\begin{enumerate}     
  \item $xyz$ 
  \item $0$
  \item $1$ 
  \item none of these
  \end{enumerate}
\end{multicols}
%
\item If $\log_e(a+c), \log_e(a-c), \log_e(a-2b+c)$ are in AP, then \hfill (1994)
    \begin{multicols}{2}
%        
\begin{enumerate}    
        \item $a, b, c$ are in AP
        \item $a^2, b^2, c^2$ are in AP
        \item $a, b, c$ are in GP
        \item $a, b, c$ are in HP
    \end{enumerate}
    \end{multicols}
\item Let $\alpha$, $\beta$ be the roots of $x^2-x+p=0$ and $\gamma$, $\delta$ be the roots of $x^2-4x+q=0$. If $\alpha$, $\beta$, $\gamma$, $\delta$ are in GP, then the integral values of $p$ and $q$ respectively are \hfill(2001)
            \begin{multicols}{4}
\begin{enumerate}    
                \item $-2, -32$
                \item $-2, 3$
                \item $-6, 3$
                \item $6, -32$
%        
    \end{enumerate}
    \end{multicols}
\item Let the positive numbers $a, b, c, d$ be in AP. Then $abc, abd, acd, bcd$ are \hfill(2001)
    \begin{multicols}{2}
\begin{enumerate}    
        \item NOT in AP/GP/H.P
        \item in AP
        \item in GP
        \item  in HP
        \end{enumerate}
\end{multicols}
\item Suppose $a, b, c$ are in AP and $a^2, b^2, c^2$ are in GP if $a<b<c$ and $a+b+c=3/2$,  then the value of $a$ is \hfill(2002)
            \begin{multicols}{4}
\begin{enumerate}    
             \item $\frac{1}{2\sqrt{2}}$
             \item $\frac{1}{2\sqrt{3}}$
             \item $\frac{1}{2}-\frac{1}{\sqrt{3}}$
             \item $\frac{1}{2}-\frac{1}{\sqrt{2}}$
            \end{enumerate}
            \end{multicols}
\item In the quadratic equation $ax^2+bx+c=0, \triangle=b^2-4ac$ and $\alpha$+$\beta$, $\alpha^2$+$\beta^2$, $\alpha^3$+$\beta^3$ are in GP where $\alpha$, $\beta $ are roots of $ax^2+bx+c=0$, then \hfill(2005)
                \begin{multicols}{4}
\begin{enumerate}    
                    \item $\triangle\neq0$
                    \item $b\triangle=0$
                    \item $c\triangle=0$
                    \item $\triangle=0$
                \end{enumerate}
                \end{multicols}
      \item  Let $ a, b, c, d $ be real numbers in GP. If $u, v, w$ satisfy the system of equations  
%    
	      \hfill \brak{1999 }
%      
\begin{align*}
	u+2v+3w&=6  
	  \\
	  4u+5v+6w&=12 
	  \\
	  6u+9v&=4 
\end{align*}
      then show that the roots of the equations 
		$$\brak{\frac{1}{u}+\frac{1}{v}+\frac{1}{w}}x^{2}+\sbrak{\brak{b-c}^{2}+\brak{c-a}^{2}+\brak{d-b}^{2}}x+u+v+w=0 $$ and $$ 20x^{2}+10\brak{a-d}^{2}x-9=0 $$ are reciprocals of each other.
 \item For any three positive real numbers $a, b$ and $c$,  $$9(25a^2+b^2)+ 25(c^2-3ac) = 15b(3a+c).$$ Then \hfill (2017)
	 \begin{multicols}{2}
\begin{enumerate}    
  \item$a, b$ and $c$ are in AP
  \item{$b, c$ and $a$ are in GP}
  \columnbreak
  \item $b, c$ and $a$ are in AP
  \item{$a, b$ and $c$ are in GP}
  \end{enumerate}
  \end{multicols}
\item       Let $a_1, a_2, a_3, \dots $ be a sequence of positive integers in arithmetic progression with common difference 2. Also, let $b_1, b_2, b_3, \dots $ be a sequence of positive integers in geometric progression with common ratio 2. If $a_1 = b_1 = c$, then the number of all possible values of $c$ for which the equality 

\begin{align*}
     2(a_1 + a_2 + \dots + a_n) = b_1 + b_2 + \dots + b_n
\end{align*}
    holds for some positive integer $n$, is \rule{1cm}{0.1pt}.
\hfill \brak{2020}
\item If the system of linear equations
\begin{align*}
    2x + 2ay + az &= 0, \\
    2x + 3by + bz &= 0, \\
    2x + 4cy + cz &= 0,
\end{align*}
where $a, b, c$ are non-zero and distinct, has a non-zero solution, then
\hfill \brak{2020}
\begin{multicols}{2}
\begin{enumerate}
    \item $a, b, c$ are in A.P.
    \item $a + b + c = 0$
    \item $a, b, c$ are in G.P.
    \item $\frac{1}{a}, \frac{1}{b}, \frac{1}{c}$ are in A.P.
\end{enumerate}
\end{multicols}
\item If the variance of the first $n$ natural numbers is 10 and the variance of the first $m$ even natural numbers is 16, then $m+n$ is equal to \rule{1cm}{0.1pt}.
\hfill \brak{2020}
\end{enumerate}


