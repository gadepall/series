\begin{enumerate}[label=\thesubsection.\arabic*, ref=\thesubsection.\theenumi]
\item In an AP,  If $p^{th}$ term is $\frac{1}{q}, q^{th}$ term is $\frac{1}{p}$,  prove that the sum of first $pq$ terms is $\frac{1}{2}(pq+1)$,  where $p \neq q.$
\item Sum of the first $p,  q$ and $r$ terms of an AP are $a,  b$ and $c$,  respectively. Prove that
$$\frac{a}{p}(q-r)+\frac{b}{q}(r-p)+\frac{c}{r}(p-q) = 0$$
\item Find the sum to $n$ terms of the AP,  whose $k^{th}$ term is $5k + 1$.
\item If the sum of $n$ terms of an AP is $pn + qn^2$,  where $p$ and $q$ are constants,  find the common difference.
\item The sums of $n$ terms of two arithmetic progressions are in the ratio $5n + 4 : 9n + 6$. Find the ratio of their $18^{th}$ terms.
\item If the sum of first $p$ terms of an AP is equal to the sum of the first $q$ terms,  then find the sum of the first $p + q$ terms.
\item The ratio of the sums of $m$ and $n$ terms of an AP is $m^2 : n^2$. Show that the ratio of 
$m^{th}$ and $n^{th}$ term is $(2m - 1) : (2n - 1)$.
\item If the sum of $n$ terms of an AP is $3n^2 + 5n$ and its $m^{th}$ term is 164,  find the value
of $m$.
\item Show that the sum of $(m + n)^{th}$ and $(m - n)^{th}$ terms of an AP is equal to twice the $m^{th}$ term.
\item Let the sum of $n,  2n,  3n$ terms of an AP be $y\brak 1,  y\brak 2$ and $y\brak 3$,  respectively,  show that 
$$y\brak 3 = 3(y\brak 2 - y\brak 1)$$
\item The $p^{th}, q^{th}$ and $r^{th}$ terms of an AP  are $a, b, c,$ respectively. Show that 
$$\brak {q - r }a + \brak {r - p }b + \brak {p - q }c = 0.$$
\item If $$a\brak {\frac{1}{b}+\frac{1}{c}}, b\brak {\frac{1}{c}+\frac{1}{a}}, c\brak {\frac{1}{a}+\frac{1}{b}}$$ are in AP, prove that $a, b, c$ are in AP. 
\item In an AP if the $m^{th}$ is $n$ and the $n^{th}$ term is $m$, where $m \ne n$, find the $p^{th}$ term.
\item If the sum of $n$ terms of an AP is
	$$nP+\frac{1}{2} n\brak {n-1}Q,$$
	where $P$ and $Q$ are constants, find the common difference.
\item If $\frac{a^n+b^n}{a^{n-1}+b^{n-1}}$ is the AM  between $a$ and $b$,  then find the value of $n$.
\item If the sum of the first $n$ terms of an AP is $4n - n^2$,  what is the first term (that is $y\brak 1$ )? What
is the sum of first two terms? What is the second term? Similarly,  find the 3rd,  the 10th and
the $n$th terms.
\item Between 1 and 31,  $m$ numbers have been inserted in such a way that the resulting sequence is an AP and the ratio of $7^{th}$ and $(m - 1)^{th}$ numbers is 5 : 9. Find the value of $m$.
\item The $5^{th}$, $8^{th}$ and $11^{th}$ terms of a GP  are $p, q$ and $s$, respectively. Show that 
$q^2 = ps$.
\item If the $4^{th}$, $10^{th}$ and $16^{th}$ terms of a GP  are $x, y$ and $z$ respectively, prove that $x, y, z$ are in GP.
\item Show that the products of the corresponding terms of the sequences $a, ar, ar^2,\dots,  ar^{n-1}$ and $A, AR, AR^2,\dots,  AR^{n -1}$ form a GP  and find the common ratio.
\item If the $p^{th}, q^{th}$ and $r^{th}$ terms of a GP  are $a, b$ and $c$, respectively. Prove that 
$$a^{q-r} b^{r-p} c^{p-q} = 1.$$
\item If the first and the $n^{th}$ term of a GP  are $a$ and $b$, respectively, and if $P$ is the product of $n$ terms, prove that $P^2 = \brak{ab}^n.$
\item If $a, b, c$ and $d$ are in GP  show that $\brak{a^2 + b^2 + c^2}\brak{b^2 + c^2 + d^2} = \brak{ab + bc + cd}^2.$
\item If $f$ is a function satisfying $$f\brak{x +y} = f\brak{x} f\brak{y} \forall x,y \in N$$ such that 
	$$f\brak{1} = 3\text{ and }\sum_{x = 1}^n f\brak{x} = 120,$$ find the value of $n$.
	\\
	\solution Since 
\begin{align}
	f\brak{2}=
	f\brak{1}
	f\brak{1}
	=
	\sbrak{f\brak{1}}^2,
\end{align}
it is easy to verify that
\begin{align}
	f\brak{k}&=
	\sbrak{f\brak{1}}^k
	\\
	\therefore
\sum_{k = 1}^n f\brak{k} 
&=
\sum_{k = 1}^n 3^{k} 
	\\
	\frac{3\brak{3^{n}-1}}{2} &= 120,
	\\
	\text{or, }
	n =\frac{\log 81}{\log 3} &= 4
\end{align}
\item A GP  consists of an even number of terms. If the sum of all the terms is 5 times the sum of terms occupying odd places, then find its common ratio.
\item  The sum of the first four terms of an AP  is 56. The sum of the last four terms is 112. If its first term is 11, then find the number of terms.
\item If $$\frac{a+bx}{a-bx} = \frac{b+cx}{b-cx} = \frac{c+dx}{c-dx}\brak{x \neq 0},$$ then show that $a, b, c$ and $d$ are in GP.
\item Let $S$ be the sum, $P$ the product and $R$ the sum of reciprocals of $n$ terms in a GP.  Prove that $P^2R^n = S^n.$ 
\item The $p^{th}, q^{th}$ and $r^{th}$ terms of an AP  are $a, b, c,$ respectively. Show that 
$$\brak{q - r }a + \brak{r - p }b + \brak{p - q }c = 0.$$
\item If $$a\brak{\frac{1}{b}+\frac{1}{c}}, b\brak{\frac{1}{c}+\frac{1}{a}}, c\brak{\frac{1}{a}+\frac{1}{b}}$$ are in AP, prove that $a, b, c$ are in AP 
\item If $a, b, c, d$ are in GP  prove that $$\brak{a^n + b^n}, \brak{b^n + c^n}, \brak{c^n + d^n}$$ are in GP. 
\item If $a$ and $b$ are the roots of $$x^2 - 3x + p = 0$$ and $c, d$ are roots of $$x^2 - 12x + q = 0,$$ where $a, b, c, d$ form a GP,  prove that $$\brak{q + p} : \brak{q - p} = 17:15.$$
\item The ratio of the AM  and GM  of two positive numbers $a$ and $b$, is $m : n$. Show that 
$$a:b = \brak{m+\sqrt{m^2 - n^2}}:\brak{m - \sqrt{m^2 - n^2}}.$$
\item If $a, b, c$ are in AP; $b, c, d$ are in GP  and $$\frac{1}{c}, \frac{1}{d}, \frac{1}{e}$$ are in AP  prove that $a, c, e$ are in GP.
\item Show that the ratio of the sum of first $n$ terms of a GP  to the sum of terms from $\brak{n+1}^{th}$ to $\brak{2n}^{th}$ term is $\frac{1}{r^n}$.
\item Insert two numbers between 3 and 81 so that the resulting sequence is GP 
\item Find the value of $n$ so that $$\frac{a^{n + 1} + b^{n + 1}}{a^n + b^n}$$ may be the geometric mean between $a$ and $b$.
\item The sum of two numbers is 6 times their geometric mean, show that numbers are in the ratio $$\brak{3+2\sqrt{2}}:\brak{3-2\sqrt{2}}.$$
\item If $A$ and $G$ be AM  and GM, respectively between two positive numbers, prove that the numbers are $$A \pm \sqrt{\brak{A+G}\brak{A-G}}.$$
\item If the $p^{th}, q^{th}, r^{th}$ and $s^{th}$ terms of an AP are in GP, then show that $\brak{p-q},\brak{q-r},\brak{r-s}$ are also in GP.
\item If $a,b,c$ are in GP and $a^{\frac{1}{x}}b^{\frac{1}{y}}c^{\frac{1}{z}}$, prove that $x,y,z$ are in AP.
\item If $a, b, c, d, p$ are different real numbers such that 
	$$\brak{a^2+b^2+c^2}p^2-2\brak{ab+bc+cd}p+\brak{b^2+c^2+d^2} \le 0,$$ then show that $a,b,c$ and $d$ are in GP.
\item If $p,q,r$ are in GP and the equations 
	$$px^2+qx+r = 0, \quad dx^2+2ex+f = 0$$
	have a common root, then show that
	$$\frac{d}{p},\frac{e}{q},\frac{f}{r}$$ are in AP.
\end{enumerate}
